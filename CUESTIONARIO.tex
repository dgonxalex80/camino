% Options for packages loaded elsewhere
\PassOptionsToPackage{unicode}{hyperref}
\PassOptionsToPackage{hyphens}{url}
%
\documentclass[
  12pt,
]{article}
\usepackage{amsmath,amssymb}
\usepackage{iftex}
\ifPDFTeX
  \usepackage[T1]{fontenc}
  \usepackage[utf8]{inputenc}
  \usepackage{textcomp} % provide euro and other symbols
\else % if luatex or xetex
  \usepackage{unicode-math} % this also loads fontspec
  \defaultfontfeatures{Scale=MatchLowercase}
  \defaultfontfeatures[\rmfamily]{Ligatures=TeX,Scale=1}
\fi
\usepackage{lmodern}
\ifPDFTeX\else
  % xetex/luatex font selection
    \setmainfont[]{Architects Daughter}
\fi
% Use upquote if available, for straight quotes in verbatim environments
\IfFileExists{upquote.sty}{\usepackage{upquote}}{}
\IfFileExists{microtype.sty}{% use microtype if available
  \usepackage[]{microtype}
  \UseMicrotypeSet[protrusion]{basicmath} % disable protrusion for tt fonts
}{}
\makeatletter
\@ifundefined{KOMAClassName}{% if non-KOMA class
  \IfFileExists{parskip.sty}{%
    \usepackage{parskip}
  }{% else
    \setlength{\parindent}{0pt}
    \setlength{\parskip}{6pt plus 2pt minus 1pt}}
}{% if KOMA class
  \KOMAoptions{parskip=half}}
\makeatother
\usepackage{xcolor}
\usepackage[letterpaper, top=1.5cm, bottom=1.5cm, left=2cm,
right=2cm]{geometry}
\usepackage{graphicx}
\makeatletter
\def\maxwidth{\ifdim\Gin@nat@width>\linewidth\linewidth\else\Gin@nat@width\fi}
\def\maxheight{\ifdim\Gin@nat@height>\textheight\textheight\else\Gin@nat@height\fi}
\makeatother
% Scale images if necessary, so that they will not overflow the page
% margins by default, and it is still possible to overwrite the defaults
% using explicit options in \includegraphics[width, height, ...]{}
\setkeys{Gin}{width=\maxwidth,height=\maxheight,keepaspectratio}
% Set default figure placement to htbp
\makeatletter
\def\fps@figure{htbp}
\makeatother
\setlength{\emergencystretch}{3em} % prevent overfull lines
\providecommand{\tightlist}{%
  \setlength{\itemsep}{0pt}\setlength{\parskip}{0pt}}
\setcounter{secnumdepth}{-\maxdimen} % remove section numbering
\usepackage{amsmath}
\usepackage{amssymb}
\ifLuaTeX
  \usepackage{selnolig}  % disable illegal ligatures
\fi
\usepackage{bookmark}
\IfFileExists{xurl.sty}{\usepackage{xurl}}{} % add URL line breaks if available
\urlstyle{same}
\hypersetup{
  pdftitle={Ejemplo con fuente manuscrita},
  hidelinks,
  pdfcreator={LaTeX via pandoc}}

\title{Ejemplo con fuente manuscrita}
\author{}
\date{\vspace{-2.5em}}

\begin{document}
\maketitle

\textbf{CUESTIONARIO - Convivencia de Inicio de curso  2024-2025}

Este inicio de curso 2024-2025 no pone delante de dos acontecimientos
muy importantes para la vida de la Iglesia : el Sínodo y sobre todo, el
Jubileo.

El Cardenal San \textbf{Jhon Henry Newman} (Londres 1801-1890),
convertido del anglicanismo al catolicismo, en una conferencia sobre la
misión profética de la Iglesia, titulada :
``\textbf{La pasión de Cristo continúa en su Iglesia}'' (Conf.14)
explica:

En verdad, cuando analizamos toda la
\textbf{historia del cristianismo desde sus orígenes}, descubrimos que
\textbf{no es más que una serie de problemas y de desórdenes}.

Cada siglo es igual a los demás, pero para quien lo vive, le parece peor
que todos los tiempos anteriores. La Iglesia está siempre enferma y
permanentemente débil, llevando siempre en su cuerpo la muerte de Jesús,
para que también la vida de Jesús se manifieste en su cuerpo.

Los profetas tienen razón al expresar: ``¿Cuándo terminarán estas cosas
maravillosas, Señor? ¿Cuanto durará este misterio?.
\textbf{Solo Dios conoce el día y la hora} en que se cumplirá lo que
debe suceder\ldots{} . Mientras tanto, nos consuela contemplar lo que ha
sucedido en el pasado, para que
\textbf{no nos desesperemos no nos desanimemos ni nos angustiemos por los problemas que nos rodean}.
Siempre ha habido problemas y siempre habrá problemas; son nuestra
herencia.

El Papa Francisco en la Bula de convocatoria del Jubileo de 2025, lo
presenta como el \textbf{Jubileo de la esperanza}, ``Spes non
Confidundit'' (Rom 5,5).

San Pablo escribe : ``Nos gloriamos incluso en las tribulaciones,
sabiendo que la tribulación produce paciencia, la paciencia virtud
probada y la virtud probada esperanza''. (Rom 5,3-4). Para el Apóstol,
la tribulación y el sufrimiento son las condiciones típicas de los que
anuncian el Evangelio en contexto de incomprensión y de persecución
(cf.~2Co 6,3-10). Pero en tales situaciones a través de la oscuridad se
vislumbra una luz: se descubre cómo hay una fuerza que sustenta la
evangelización que brota de la cruz y de la resurrección de Cristo. Esto
lleva a desarrollar una virtud estrechamente relacionada con la
esperanza: la paciencia \ldots{} De este entrelazado de esperanza y
paciencia aparece de forma clara, cómo la vida cristiana es un camino,
que necesita también momentos fuertes para nutrir y fortalecer la
esperanza, compañera insustituible que deja entrever la meta : el
encuentro con el Señor Jesús'' (nn 4-5) .

``Bajo el signo de la esperanza el apóstol Pablo infunde ánimo a la
comunidad cristiana de Roma. La esperanza es también el mensaje central
de próximo Jubileo. Pienso en todos los peregrinos de la esperanza que
vendrán a Roma para vivir el Año Santo y en aquellos que no pudiendo
venir a la ciudad de los apóstoles Pedro y Pablo, lo celebrarán en las
Iglesias particulares. Que sea para todos un momento de encuentro vivo y
personal con el Señor Jesús,''puerta'' de salvación (cf.~Jn 10, 7.9) con
Él, a quien la Iglesia tiene la misión de anunciar siempre, en todas las
partes y a todos como ``nuestra esperanza (Tm 1,1)'' (n.1).

\vspace{.3cm}

\textbf{PREGUNTAS} :

\begin{itemize}
    \item[1.] El próximo Jubileo es una nueva llamada a la santidad de vida. ¿Cómo  vives tú esta llamada a la santidad personalmente, en ti familia, en el Camino Neocatecumenal? ¿Qé dificultades encuentras? ¿Encuentras ayuda en la participación en tu comunidad?
    \item[2.] En el mes de octubre se celebró el Sínodo deseado por el Papa Francisco. En tu opinión ¿cuál crees que la misión que el Señor encomienda al Camino Neocatecumenal ante los desafíos a los que se enfrenta la Iglesia en el mundo de hoy?
\end{itemize}

\end{document}
